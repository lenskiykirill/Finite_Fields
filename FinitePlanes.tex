%%%%%%%%%%%%%%%%%%%%%%%%%%%%%%%%%%%%%%%%%%%%%%%%%%%%%%%%%%%%%%%
%
% Welcome to Overleaf --- just edit your LaTeX on the left,
% and we'll compile it for you on the right. If you open the
% 'Share' menu, you can invite other users to edit at the same
% time. See www.overleaf.com/learn for more info. Enjoy!
%
%%%%%%%%%%%%%%%%%%%%%%%%%%%%%%%%%%%%%%%%%%%%%%%%%%%%%%%%%%%%%%%

% ===============================================
% MATH 790: Real Analysis           Spring 2022
% hw_template.tex
% ===============================================

% -------------------------------------------------------------------------
% The preamble that follows can be ignored. Go on
% down to the section that says "START HERE" 
% -------------------------------------------------------------------------

\documentclass{article}

\usepackage[T2A]{fontenc}
\usepackage[utf8]{inputenc}
\usepackage[russian]{babel}
\usepackage{amsmath,amsfonts}
\usepackage{tikz}
\usetikzlibrary{}
\usetikzlibrary{calc}

\usepackage[margin=1in]{geometry} 
\usepackage{amsmath,amsthm,amssymb,hyperref}

\newcommand{\R}{\mathbf{R}}  
\newcommand{\Z}{\mathbf{Z}}
\newcommand{\N}{\mathbf{N}}
\newcommand{\Q}{\mathbf{Q}}

\newenvironment{theorem}[2][Теорема]{\begin{trivlist}
\item[\hskip \labelsep {\bfseries #1}\hskip \labelsep {\bfseries #2.}]}{\end{trivlist}}
\newenvironment{definition}[2][Определение]{\begin{trivlist}
\item[\hskip \labelsep {\bfseries #1}\hskip \labelsep {\bfseries #2.}]}{\end{trivlist}}
\newenvironment{lemma}[2][Лемма]{\begin{trivlist}
\item[\hskip \labelsep {\bfseries #1}\hskip \labelsep {\bfseries #2.}]}{\end{trivlist}}
\newenvironment{exercise}[2][Упражнение]{\begin{trivlist}
\item[\hskip \labelsep {\bfseries #1}\hskip \labelsep {\bfseries #2.}]}{\end{trivlist}}
\newenvironment{problem}[2][Задача]{\begin{trivlist}
\item[\hskip \labelsep {\bfseries #1}\hskip \labelsep {\bfseries #2.}]}{\end{trivlist}}
\newenvironment{question}[2][Открытый вопрос]{\begin{trivlist}
\item[\hskip \labelsep {\bfseries #1}\hskip \labelsep {\bfseries #2.}]}{\end{trivlist}}
\newenvironment{corollary}[2][Следствие]{\begin{trivlist}
\item[\hskip \labelsep {\bfseries #1}\hskip \labelsep {\bfseries #2.}]}{\end{trivlist}}
\newenvironment{hence}[2][Следствие]{\begin{trivlist}
\item[\hskip \labelsep {\bfseries #1}\hskip \labelsep {\bfseries #2}]}{\end{trivlist}}

\newenvironment{solution}{\begin{proof}[Solution]}{\end{proof}}

\newcommand\FanoPlane[1][1cm]{%
\begin{tikzpicture}[
mydot/.style={
  draw,
  circle,
  fill=black,
  inner sep=1.2pt}
]
\draw
  (0,0) coordinate (A) --
  (#1,0) coordinate (B) --
  ($ (A)!.5!(B) ! {sin(60)*2} ! 90:(B) $) coordinate (C) -- cycle;
\coordinate (O) at
  (barycentric cs:A=1,B=1,C=1);
\draw (O) circle [radius=#1*1.717/6];
\draw (C) -- ($ (A)!.5!(B) $) coordinate (LC); 
\draw (A) -- ($ (B)!.5!(C) $) coordinate (LA); 
\draw (B) -- ($ (C)!.5!(A) $) coordinate (LB); 
\foreach \Nodo in {A,B,C,O,LC,LA,LB}
  \node[mydot] at (\Nodo) {};    
\end{tikzpicture}%
}

\begin{document}

\title{Конечные геометрии}
\author{Ленский Кирилл\\матпрактикум ФПМИ}

\maketitle

% -----------------------------------------------------
% The following two environments (theorem, proof) are
% where you will enter the statement and proof of your
% first problem for this assignment.
%
% In the theorem environment, you can replace the word
% "theorem" in the \begin and \end commands with
% "exercise", "problem", "lemma", etc., depending on
% what you are submitting. 
% -----------------------------------------------------

\begin{definition}{(Конечное поле)}
Конечным полем (полем Галуа) $GF(q)$ для некоторой степени простого $q=p^k$ называется поле из $q$ элементов
\end{definition}
\begin{exercise}{(Корректность определения)}
Докажите, что $\forall{q=p^k}$ существует $GF(q)$, причем единственное с точностью до изоморфизма.
\end{exercise}

\begin{definition}{(Геометрия Галуа)}
Пусть $V$ - векторное пространство размерности $n+1$ над полем $GF(q)$. Тогда проективное пространство $PG(n,q)$ состоит из $V'\subset V:dim V \neq 0$.
\end{definition}

\begin{exercise}{(Число подпространств)}
Докажите, что подпространств $V$ размерности $m+1$ в точности $\frac{(q^{n+1}-1)(q^{n+1}-q)\cdots(q^{n+1}-q^{m})}{(q^{m+1}-1)(q^{m+1}-q)\cdots(q^{m+1}-q^m)}$
\end{exercise}

\begin{exercise}{(Конечная проективная плоскость)}
Пусть есть множествa "точек" $P$ и "прямых" $L$, и отношение инцидентности $R\in2^{P\times L}$ такие, что существуют 4 точки, никакие 3 из которых не инцидентны одной прямой, любым двум точкам инцидентна ровно одна общая прямая, а любой паре прямых - ровно одна общая точка. Тогда

а) каждые 2 прямые инцидентны равному числу точек

б) каждые 2 точки инцидентны равному числу прямых

в) пусть на прямой лежит $q+1$ точка. Тогда через каждую точку проходит $q+1$ прямая.
% рассмотреть точку не на прямой. И прямые через нее

г) всего точек, как и прямых, $q^2+q+1$
\end{exercise}
\begin{center}
    \FanoPlane[5cm]
\end{center}

Можно было бы подумать, что любая конечная проективная плоскость изоморфна $PG(2,q)$ для некоторого поля $GF(q)$, но это не так. Контрпример существует уже при $q=9$. С другой стороны, до сих пор не выяснено, всегда ли порядок конечной проективной плоскости - степень простого числа.
% На самом деле результат не столь удивительный
% ведь даже для плоскости порядка 12 компьютер уже не
% справляется с перебором.

\begin{definition}{(Афинная геометрия Галуа)}
Для некоторого натурального $n$ и степени простого $q=p^k$ афинной плоскостью Галуа $AG(n,q)$ называется множество $PG(n,q)$, без одного из подпространств размерности $n$ и всех инцидентных ему. (Точки в $AG$ можно описать числами $(x_1,\dots, x_n), \\{\forall{i}\; x_i\in[1,q]}$)
\end{definition}

По аналогии с конечной проективной плоскостью, можно определить и \emph{конечную афинную плоскость}. Это такой набор точек и прямых, что существуют четыре точки, никакие три из которых не инцидентны одной прямой, у любых двух точек найдется общая прямая, инцидентная им обеим, а для любой прямой $l$ и точки $P$, не инцидентной ей, $\exists!$ прямая $l'$ такая, что $P$ инцидентна ей, а $l$ не имеет с ней общих инцидентных точек.

\begin{exercise}{(Двойственность афинной и проективной плоскостей)}
Если существует конечная афинная плоскость порядка $q$, то существует и проективная того же порядка. И наоборот.
\end{exercise}

\begin{hence}{}
Из решения задачи становится очевидным, что любая конечная афинная плоскость состоит из $q^2$ точек и $q^2+q$ прямых.
\end{hence}

\begin{theorem}{Дезарга}
На $\mathbb{R}\mathbf{P}^2$ теорема Дезарга формулируется так: пусть даны $\triangle ABC$, $\triangle A'B'C'$. Пусть $AA'$, $BB'$ и $CC'$ пересекаются в одной точке. Тогда точки $AB\cap A'B'$, $BC\cap B'C'$ и $CA\cap C'A'$ лежат на одной прямой. Однако в других проективных геометриях аналогичное утверждение вполне может быть неверным. Геометрии, где она выполняется, называются \emph{Дезарговыми}.
\end{theorem}

%\begin{exercise}{(Связь конечной проективной плоскости и теоремы Дезарга)*}
%Конечная проективная плоскость порядка $q$ является проективной плоскостью над некоторым телом (кольцом, в котором у каждого элемента есть обратный) тогда и только тогда, когда она Дезаргова.
%\end{exercise}

\begin{theorem}{(Tamas Szonyi, 1994)}

\end{theorem}

\begin{theorem}{(Benedetto, Solymosi, White, 2020)}
Пусть $A,B\subset GF(p),\; {|A|},{|B|}\ge 2$. Тогда число различных \emph{направлений} (классов эквивалентности параллельных прямых) во множестве вида 
\end{theorem}

\begin{proof}
\emph{Галуа крут}
\end{proof}

\end{document}
